\synctex=1
\documentclass[a4paper,11pt]{article}
\usepackage[utf8]{inputenc}
\usepackage{graphicx}
\usepackage[a4paper, total={6.5in, 9in}]{geometry}
\usepackage{natbib}
\usepackage{authblk} % author affiliations
\usepackage[iso,english]{isodate}
\usepackage{nameref}
\newcommand*{\qref}[1]{\hyperref[{#1}]{\textit{``\nameref*{#1}'' (section \ref*{#1})}}}
\newcommand*{\qrefP}[1]{\hyperref[{#1}]{\textit{``\nameref*{#1}'', section \ref*{#1}}}}
\newcommand*{\qrefS}[1]{\hyperref[{#1}]{section \textit{\ref*{#1},
      ``\nameref*{#1}''}}}
\usepackage{hyperref}
\usepackage{latexgit}

\hypersetup{
  colorlinks = true,
  citecolor=  black,
  linkcolor = {blue},
  filecolor = cyan %% controls color of external ref, if used
}
\title{EvAM-Tools: all additional documentation}
% \author{Ramon Diaz-Uriarte, Pablo Herrera-Nieto}


\author[1,2,$\dagger$]{Ramon Diaz-Uriarte}
\author[1,2]{Pablo Herrera Nieto}
\affil[1]{Dpt. of Biochemistry, School of Medicine, Universidad Autónoma de Madrid, Madrid, Spain}
\affil[2]{Instituto de Investigaciones Biomédicas `Alberto Sols'
  (UAM-CSIC), Madrid, Spain}
\affil[$\dagger$]{To whom correspondence should be addressed: \normalfont r.diaz@uam.es}

\date{\today \\ Version \gitcommithash}
\begin{document}
\maketitle



This PDF is the concatenation of three different files:

\begin{enumerate}
\item \textbf{EvAM-Tools: additional documentation:} Additional  documentation about EvAM-Tools. These include technical details (error models, predicting genotype frequencies, generation of random evam models), additional usage information for the Shiny web GUI, and a FAQ. % application help but others apply to the package or to both. 
\item \textbf{Package evamtools help:} The help files for the evamtools R package, collated as a single pdf.
\item \textbf{Using OncoSimulR to get accessible genotypes and transition matrices:}  How we can use OncoSimulR to get accessible genotypes and transition matrices for CBN (and MCCBN), OT, HESBCN, and OncoBN; this provides additional testing and makes the fitness models explicit.

    % I also clarify how our code interprets the output of HESBCN (not only as it concerns using OncoSimulR) and provide numerical examples. 
\end{enumerate}

\end{document}